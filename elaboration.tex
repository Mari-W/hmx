\documentclass[runningheads]{llncs}

\usepackage{graphicx}
\usepackage{fontspec}
\usepackage{unicode-math}
\usepackage[Latin,Greek]{ucharclasses}
\usepackage{amsmath}
\usepackage{proof}
\usepackage{minted}
\AtBeginEnvironment{minted}{%
  \renewcommand{\fcolorbox}[4][]{#4}}
\usepackage[backend=biber]{biblatex}
\addbibresource{references.bib}

\newfontfamily\substitutefont{CMU Serif}
\setTransitionsForGreek{\begingroup\substitutefont}{\endgroup}

% chktex-file 36 
% chktex-file 12

\newcommand{\hmx}{HM($X$)}
\newcommand{\hmr}{HM($\mathcal{R}$)}
\newcommand{\hmo}{HM($\mathcal{O}$)}

\title{\includegraphics[width=0.4\textwidth]{logo.png}~\\[1cm] Elaboration on \hmx:\\Type Inference with Constraint Types}
\titlerunning{\hmx: Type Inference with Constraint Types}
\institute{Chair of Programming Languages, University of Freiburg \\ \email{weidner@cs.uni-freiburg.de}}
\author{Marius Weidner}

\begin{document}

\let\oldaddcontentsline\addcontentsline{}
\def\addcontentsline#1#2#3{}
\maketitle
\def\addcontentsline#1#2#3{\oldaddcontentsline{#1}{#2}{#3}}

\begin{abstract}
  We discuss \hmx{}, a family of type systems that supports polymorphism, full type inference and constraint types. 
  \hmx{} is a extension to the Hindley-Milner type system, that itself restricts System F such that full type inference is decidable and unambiguous.
  The constraint system $X$ used in \hmx{} is left abstract and can be instantiated with arbitrary constraint systems that fulfill a set of conditions. 
  Because of this abstraction \hmx{} can be used to model and reason about many commonly used constraint-like type features. 
  Examples for constraint-like type features include subtyping, substructural types and type classes. 
  \hmx{} comes with a complete and sound type inference algorithm, as well as a soundness proof, that both are independent of the actual constraint system $X$.
  Thus, the work for proving theoretical properties and constructing a inference algorithm for new constraint-like type features in a HM setting is reduced. 
\end{abstract}

\setcounter{tocdepth}{2}
\tableofcontents
\newpage 

\section{Introduction}
\subsection{Hindley Milner: Polymorphism with Full Type Inference}
\subsection{Example: A Program with Constraint Types}
\section{\hmx{}}
\subsection{Syntax}
\begin{figure}[t]
  \centering
  \caption{Syntax}
\end{figure}
\subsection{Typing}
\begin{figure}[t]
  \centering
  \caption{Typing $(C, Γ ⊢ e : σ)$}
\end{figure}
\section{Instantiating \hmx{}}
\subsection{\hmr{}: Extension with Polymorphic Records}
\subsubsection{Extensions}
\begin{figure}[t]
  \centering
  \caption{Syntax}
\end{figure}
\begin{figure}[t]
  \centering
  \caption{Constraints}
\end{figure}
\subsubsection{Example}
\subsection{\hmo{}: Extension with Overloading}
\subsubsection{Extensions}
\begin{figure}[t]
  \centering
  \caption{Syntax}
\end{figure}
\begin{figure}[t]
  \centering
  \caption{Constraints}
\end{figure}
\subsubsection{Example}
\section{Metatheory}
\subsection{Soundness}
\subsection{Type Inference}
\section{Related Work \& Conclusion}
\subsection{Related Work}
\subsection{Conclusion}
% mention rust and haskell

\nocite{hmx}
\nocite{sts}
\nocite{atapl}
\printbibliography{}


\end{document}