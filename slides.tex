\documentclass[aspectratio=169]{beamer}
\usetheme{Boadilla}
\useoutertheme{infolines}
\setbeamertemplate{navigation symbols}{}

\usepackage{fontspec}
\usepackage{unicode-math}
\usepackage[Latin,Greek]{ucharclasses}
\usepackage{amsmath}
\usepackage{proof}
\usepackage{minted}
\AtBeginEnvironment{minted}{%
  \renewcommand{\fcolorbox}[4][]{#4}}
\usepackage[backend=biber]{biblatex}
\addbibresource{references.bib}

\title{HM($X$): Type Inference with Constraint Types $^{\text{\cite{hmx}}}$}
\institute[Uni Freiburg]{Chair of Programming Languages, University of
  Freiburg}
\author{Marius Weidner}

\newcommand{\KWLet}{\textbf{let }}
\newcommand{\KWIn}{\textbf{ in }}
\newcommand{\HMLet}{
  \infer[\text{Let}]{
    Γ ⊢ \KWLet x = e\KWIn e′ : τ′
  }{
    Γ ⊢ e : σ
    & Γ, x : σ ⊢ e′ : τ′
  }
}
\newcommand{\HMIntro}{
  \infer[∀ \ \text{Intro}]{
    Γ ⊢ e : ∀ \Vec α. τ
  }{
    Γ ⊢ e : τ
    & \Vec α ∉ free(Γ)
  }
}
\newcommand{\HMElim}{
  \infer[∀ \ \text{Elim}]{
    Γ ⊢ e : [\Vec τ/ \Vec α]τ
  }{
    Γ ⊢ e : ∀\Vec α. τ
  }
}
\newcommand{\HMXLet}{
  \infer[\text{Let}]{
    \textcolor{blue}{C,}\ Γ ⊢ \KWLet x = e\KWIn e′ : τ′
  }{
    \textcolor{blue}{C,}\ Γ ⊢ e : σ
    & \textcolor{blue}{C,}\ (Γ, x : σ) ⊢ e′ : τ′
  }
}
\newcommand{\HMXEq}{
  \infer[\text{\textcolor{blue}{Eq}}]{
    C,\ Γ ⊢ e : τ′
  }{
    C,\ Γ ⊢ e : τ
    & ⊢ (τ = τ′)
  }
}
\newcommand{\HMXIntro}{
  \infer[∀ \ \text{Intro}]{
    \textcolor{blue}{C ∧ ∃\Vec α.D,}\ Γ ⊢ e : ∀ \Vec α.D ⇒ τ
  }{
    \textcolor{blue}{C ∧ D,}\ Γ ⊢ e : τ
    & \Vec α ∉ free(Γ) \ \textcolor{blue}{∪ \ free(C)}
  }
}
\newcommand{\HMXElim}{
  \infer[∀ \ \text{Elim}]{
    \textcolor{blue}{C,} \ Γ ⊢ e : [\Vec τ / \Vec α] τ′
  }{
    \textcolor{blue}{C,}\ Γ ⊢ e : ∀ \Vec α.D ⇒ τ′
    & \textcolor{blue}{C ⊢ᵉ [\Vec τ / \Vec α]D}
  }
}

\begin{document}

\begin{frame}
  \titlepage
\end{frame}

\begin{frame}{Outline}
  \tableofcontents
\end{frame}

\section{HM: Polymorphic Lambda Calculus with Full Type Inference}

\begin{frame}[fragile]
  \frametitle{Hindley Milner Basics}
  \begin{columns}
    \begin{column}{0.38\textwidth}
      \begin{center}
        \begin{itemize}
          \item Let Polymorphism
          \item Full Type Inference
          \item Principle Type Property
                \begin{itemize}
                  \item most general type is inferred
                \end{itemize}
        \end{itemize}
      \end{center}
    \end{column}
    \begin{column}{0.6\textwidth}
      \begin{center}
        \begin{block}{Example: Polymorphic Identity Function}
          \begin{minted}[escapeinside=||, mathescape = true]{ocaml}
let id = |$\symbf λ$|x. x in    : |$\textcolor{blue}{\symbf{∀α. α → α}}$| 
id 42                : Int
id "Hello World"     : Str
          \end{minted}
        \end{block}
      \end{center}
      \begin{center}
        \begin{block}{Example: Polymorphic List Constructors}
          \begin{minted}[escapeinside=||, mathescape = true]{ocaml}
let nil = ..  : |$\textcolor{blue}{\symbf{∀α. α →}}$| |\textcolor{blue}{\textbf{List}}| |$\textcolor{blue}{\symbf{α}}$| 
let cons = .. : |$\textcolor{blue}{\symbf{∀α. α → α →}}$| |\textcolor{blue}{\textbf{List}}| |$\textcolor{blue}{\symbf{α →}}$| |\textcolor{blue}{\textbf{List}}| |$\textcolor{blue}{\symbf{α}}$| 
          \end{minted}
        \end{block}
      \end{center}
    \end{column}
  \end{columns}
\end{frame}

\begingroup
\Large
\begin{frame}[fragile]
  \frametitle{Hindley Milner Syntax}
  \begin{columns}
    \begin{column}{0.45\textwidth}
      \begin{align*}
        e \ ∷= & \ \ x                    \\
               & | \ λx. e                \\
               & | \ e \ e                \\
               & | \ \KWLet x = e \KWIn e
      \end{align*}
    \end{column}
    \begin{column}{0.45\textwidth}
      \begin{center}
        \begin{align*}
          τ \ ∷= & \ \ α     \\
                 & | \ τ → τ
        \end{align*}
      \end{center}
      \begin{center}
        \begin{align*}
          σ \ ∷= & \ \ τ     \\
                 & | \ ∀ \Vec α . τ
        \end{align*}
      \end{center}
    \end{column}
  \end{columns}
\end{frame}
\endgroup

\begingroup
\Large
\begin{frame}[fragile]
  \frametitle{Hindley Milner Typing}
  \begin{columns}
    \begin{column}{0.45\textwidth}
      \begin{center}
        \HMLet
      \end{center}
      % \begin{center}
      %   \begin{block}{Example: Instance Relation}
      %     $∀α. α → α \ \ ⊑ \ \ $ Int $→$ Int
      %   \end{block}
      % \end{center}
    \end{column}
    \begin{column}{0.45\textwidth}
      \begin{center}
        \HMIntro
      \end{center}
      \begin{center}
        \HMElim
      \end{center}
    \end{column}
  \end{columns}
\end{frame}
\endgroup

\begingroup
\large
\begin{frame}[fragile]
  \frametitle{Outlook: A Program with Constraint Types}
  \begin{center}
    \begin{block}{Example: Record Subtyping Constraints}
      \begin{minted}[escapeinside=||, mathescape = true]{ocaml}
let max = |$\symbf λ$|r|$₁$|. |$\symbf λ$|r|$₂$|.           : |$\textcolor{blue}{\symbf{∀α.}}(\textcolor{blue}{\symbf{α}} ≤$|{key:|\textcolor{blue}{\textbf{Int}}|}|$)\textcolor{blue}{\symbf{∀β.}}(\textcolor{blue}{\symbf{β}} ≤$|{key:|\textcolor{blue}{Int}|}|$) \ \textcolor{blue}{\symbf{ ⇒}}$|
  r|$₁$|.key |$≥$| r|$₂$|.key in             |$\textcolor{blue}{\symbf{α → β →}}$| Bool
max {key = 17, foo = "bar"}   : Bool
    {key = 42, bar = "foo"}
          \end{minted}
    \end{block}
  \end{center}
\end{frame}
\endgroup

\section{Constraint Systems: Formulate Constraints on Types}

\begingroup
\Large
\begin{frame}[fragile]
  \frametitle{Constraint Syntax}
  \begin{align*}
    C \ ∷= & \ \ ⊥     \\
           & | \ ⊤     \\
           & | \ C ∧ C \\
           & | \ ∃α. C \\
           & | \ τ = τ \\
           & | \ ⋯
  \end{align*}
\end{frame}
\endgroup

\begin{frame}[fragile]
  \frametitle{Constraint Conditions}
  \begin{columns}
    \begin{column}{0.45\textwidth}
      \begin{block}{Entailment}
        \setlength\abovedisplayskip{0pt}
        \begin{align*}
          C ⊢ D & \text{ iff } C \text{ implies } D \\
          C = D & \text{ iff } C ⊢ D ∧ D ⊢ C        \\
          ⊢C    & \text{ iff } ⊤ ⊢ C
        \end{align*}
      \end{block}
      \begin{block}{Existential Quantification}
        \setlength\abovedisplayskip{0pt}
        \begin{align*}
          C             & ⊢ ∃α. C                      \\
          C ⊢ D         & \text{ implies } ∃α.C ⊢ ∃α.D \\
          ∃α.(C ∧ ∃α.D) & = ∃α.C ∧ ∃α.D                \\
          ∃α. ∃β.C      & = ∃β. ∃α.C
        \end{align*}
      \end{block}
    \end{column}
    \begin{column}{0.48\textwidth}
      \begin{block}{Equality}
        \setlength\abovedisplayskip{0pt}
        \begin{align*}
                            & ⊢ (α = α) \\
          (α = β)           & ⊢ (β = α) \\
          (α = β) ∧ (β = γ) & ⊢ (α = γ)
        \end{align*}
      \end{block}
      \begin{block}{Substitution}
        \setlength\abovedisplayskip{0pt}
        \begin{align*}
          (τ = τ′)                & ⊢ (Τ[τ] = Τ[τ′])                      \\
          [τ/α]C = ∃α.C ∧ (α = τ) & \text{ for } α \text{ not free in } τ
        \end{align*}
      \end{block}
    \end{column}
  \end{columns}
\end{frame}

\begingroup
\large
\begin{frame}[fragile]
  \frametitle{Solved Form}
  \begin{columns}
    \begin{column}{0.6\textwidth}
      \begin{block}{Satisfiability}
        \setlength\abovedisplayskip{0pt}
        \begin{align*}
          sat(C) \text{ iff } ⊢ ∃\Vec α.C \text{ for } \Vec α \text{ free in }
          C
        \end{align*}
      \end{block}
      \begin{block}{Constraints in Solved Form}
        \setlength\abovedisplayskip{0pt}
        \begin{align*}
          C & ∈ 𝓢 \text{ implies } sat(C)                               \\
          C & ∈ 𝓢 \text{ and } C ⊢ (τ = τ′) \text{ implies } ⊢ (τ = τ′) \\
          C & ∈ 𝓢 \text{ implies } ∃α.C ∈ 𝓢
        \end{align*}
      \end{block}
    \end{column}
  \end{columns}
\end{frame}
\endgroup

\begingroup
\Large
\section{HM($X$): HM Parametrized over Constraint Systems $X$}
\begin{frame}[fragile]
  \frametitle{HM($X$) Syntax}
  \begin{columns}
    \begin{column}{0.45\textwidth}
      \begin{center}
        \begin{align*}
          e \ ∷= & \ x                      \\
                 & | \ λx. e                \\
                 & | \ e \ e                \\
                 & | \ \KWLet x = e \KWIn e
        \end{align*}
      \end{center}
    \end{column}
    \begin{column}{0.45\textwidth}
      \begin{center}
        \begin{align*}
          τ \ ∷= & \ α       \\
                 & | \ τ → τ
        \end{align*}
      \end{center}
      \begin{center}
        \begin{align*}
          σ \ ∷= & \ τ                                                    \\
                 & | \ ∀\Vec α.\textcolor{blue}{C ⇒} τ \text{ for } C ∈ 𝓢
        \end{align*}
      \end{center}
    \end{column}
  \end{columns}
\end{frame}
\endgroup

\begin{frame}[fragile]
  \frametitle{HM($X$) Typing}
  \begin{columns}
    \begin{column}{0.45\textwidth}
      \begin{center}
        \HMXLet
      \end{center}
      \begin{center}
        \HMXEq
      \end{center}
    \end{column}
    \begin{column}{0.45\textwidth}
      \begin{center}
        \HMXIntro
      \end{center}
      \begin{center}
        \HMXElim
      \end{center}
    \end{column}
  \end{columns}
\end{frame}

\section{HM($𝓡$): Instantiating HM($X$) with Polymorphic Records}

\begingroup
\Large
\begin{frame}[fragile]
  \frametitle{HM($𝓡$) Syntax}
  \begin{columns}
    \begin{column}{0.45\textwidth}
      \begin{center}
        \begin{align*}
          e \ ∷= & \ \dots                            \\
                 & | \ \{lᵢ = eᵢ\} \text{ for } i ∈ ℕ \\
                 & | \ e.l
        \end{align*}
      \end{center}
    \end{column}
    \begin{column}{0.45\textwidth}
      \begin{center}
        \begin{align*}
          τ \ ∷= & \ \dots                            \\
                 & | \ \{lᵢ : τᵢ\} \text{ for } i ∈ ℕ
        \end{align*}
      \end{center}
    \end{column}
  \end{columns}
  \begin{center}
    \begin{align*}
      C \ ∷= & \ \dots            \\
             & | \ τ ≤ \{l : τ′\}
    \end{align*}
  \end{center}
\end{frame}

\endgroup
\begin{frame}[fragile]
  \frametitle{HM($𝓡$) Constraint Conditions}
  \begin{columns}
    \begin{column}{0.5\textwidth}
      \begin{block}{Subtyping}
        \setlength\abovedisplayskip{0pt}
        \begin{align*}
          ⊢ \{l₁ : τ₁, .., lₙ : τₙ\}           & ≤ \{lᵢ : τᵢ\} \\
          τ ≤ \{l : τ₁\} ∧ τ ≤ \{l : τ₂\}      & ⊢ τ₁ = τ₂     \\
          \{.., l : τ₁, ..\} ≤ \{l : τ₂\}      & ⊢ τ₁ = τ₂     \\
          \{l₁ : τ₁, .., lₙ : τₙ\} ≤ \{l : τ\} & ⊢ ⊥
        \end{align*}
      \end{block}
      \begin{block}{Restrict Recusion}
        \setlength\abovedisplayskip{0pt}
        \begin{align*}
          ∃α. α ≤ \{l : τ\} & = ⊤ \text{ for } α \text{ not free in } τ \\
          ∃α. α ≤ \{l : τ\} & = ⊥ \text{ for } α \text{ free in } τ
        \end{align*}
      \end{block}
    \end{column}
    \begin{column}{0.4\textwidth}
      \begin{block}{Ordering}
        \setlength\abovedisplayskip{0pt}
        \begin{align*}
          ⊢ & \{lᵢ : τᵢ\} = \{l_{π(i)} : τ_{π(i)}\}   \\
            & \text{ where } π \text{ is a permutation}
        \end{align*}
      \end{block}
      \begin{block}{Solved Form}
        \setlength\abovedisplayskip{0pt}
        \begin{align*}
          α &≤ \{l : τ\} ∈ 𝓢 \\ 
          & \text{ when } α ≤ \{l : τ\} \text{ non-recursive}
        \end{align*}
      \end{block}
    \end{column}
  \end{columns}
\end{frame}

\begin{frame}[fragile]
  \frametitle{HM($𝓡$) Advanced Example}
  \begin{columns}
    \begin{column}{0.85\textwidth}
      \begin{center}
        \begin{block}{Example: Nested Record Subtyping Constraints}
          \begin{minted}[escapeinside=||, mathescape = true]{ocaml}
let unwrap = |$\symbf λ$|r.        : |$\textcolor{blue}{\symbf{∀α.∃β}}(\textcolor{blue}{\symbf{α}} ≤$|{outer:|$\textcolor{blue}{\symbf{β}}$|}|$ ∧ \textcolor{blue}{\symbf{β}} ≤$|{inner:|\textcolor{blue}{Int}|}|$) \ \textcolor{blue}{\symbf{ ⇒}}$|
  r.outer.inner in         |$\textcolor{blue}{\symbf{α → }}$| Int
unwrap {                : Int
  outer = {
    inner = 42, 
    bar = "foo"
  }, 
  foo = "bar"
} 
          \end{minted}
        \end{block}
      \end{center}
    \end{column}
  \end{columns}
\end{frame}

\section{Meta Theory \& Conclusion: Properties of HM($X$) and beyond HM($X$)}

\begingroup
\large
\begin{frame}[fragile]
  \frametitle{Meta Theoretical Properties \& Outlook}
  \begin{itemize}
    \item Full Type Inference: HM($X$) comes with a type inference algorithm
          that
          \begin{itemize}
            \item solves typing problems by translating them to constraint
                  problems
            \item preserves the \emph{principal type property}, if $X$ fulfills
                  the \emph{principal constraint property}
          \end{itemize}
    \item Type Soundness: HM($X$) is sound, i.e. all valid typing judgements
          \\ \quad $C, Γ ⊢ e : σ$ where $sound(C)$ imply \\
          \quad $e ↪ e′ ∨ val(e)$ (\emph{progress}) and
          $e ↪ e′ → C, Γ ⊢ e′ : σ$ (\emph{subject reduction})
          $^{\text{\cite{sts}}}$
    \item SHM($X$): extension with subtyping constraint $τ ≤ τ′$, \\
          \quad small extension to type inference algorithm required
  \end{itemize}
\end{frame}
\endgroup

\begin{frame}[fragile]
  \frametitle{References}
  \nocite{atapl}
  \printbibliography
\end{frame}

\end{document}